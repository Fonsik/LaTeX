\documentclass[10pt,a4paper]{article}
\usepackage[utf8]{inputenc}
\usepackage{amsmath}
\usepackage{gensymb}
\usepackage{amsfonts}
\usepackage{siunitx}
\usepackage[european]{circuitikz}
\usepackage{geometry}
\newgeometry{tmargin=2cm, bmargin=2cm, lmargin=2cm, rmargin=2cm}
\usepackage{amssymb}
\usepackage{polski}
\usepackage{graphicx}
\author{\textbf{T. Fąs}}
\title{\textbf{PRZEWODNICTWO CIEPLNE MIEDZI}}
\begin{document}
\maketitle

\begin{center}
\textbf{\subsection*{STRESZCZENIE}}
\end{center}
Celem doświadczenia było wyznaczenie współczynnika przewodnictwa cieplnego miedzi $\lambda$. Wartość otrzymana w eksperymencie wynosi $\lambda=1621\pm27$ W/mK, która to jest czterokrotnie większa od wartości tablicowych. Nie udało się ustalić przyczyn takiego odchylenia.
\begin{center}
\textbf{\subsection*{WSTĘP}}
\end{center}


\begin{center}
\textbf{\subsection*{UKŁAD DOŚWIADCZALNY}}
\end{center}

\begin{center}
\textbf{\subsection*{WYNIKI POMIARÓW}}
\end{center}


\begin{center}
\textbf{\subsection*{ANALIZA DANYCH}}
\end{center}


\begin{center}
\textbf{\subsection*{DYSKUSJA WYNIKÓW I WNIOSKI}}
\end{center} 