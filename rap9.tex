\documentclass[10pt,a4paper]{article}
\usepackage[utf8]{inputenc}
\usepackage{amsmath}
\usepackage{gensymb}
\usepackage{amsfonts}
\usepackage{siunitx}
\usepackage[european]{circuitikz}
\usepackage{geometry}
\newgeometry{tmargin=2cm, bmargin=2cm, lmargin=2cm, rmargin=2cm}
\usepackage{amssymb}
\usepackage{polski}
\usepackage{graphicx}
\author{\textbf{T. Fąs}}
\title{\textbf{DOŚWIADCZALNE SPRAWDZENIE DRUGIEJ ZASADY DYNAMIKI}}
\begin{document}
\maketitle


\begin{center}
\textbf{\subsection*{STRESZCZENIE}}
\end{center}
Celem doświadczenia było sprawdzenie, czy w warunkach eksperymentu spełniona jest druga zasada dynamiki Newtona. Udało się udowodnić, że przyśpieszenie jest wprost proporcjonalne do działającej siły oraz odwrotnie proporcjonalne do masy ciała. Dodatkowo, korzystając z wyznaczonej doświadczalnie masy układu, wyznaczono masę efektywną bloczka $m_{eff}=0,051\pm0,071$ kg.


\begin{center}
\textbf{\subsection*{WSTĘP}}
\end{center}




\begin{center}
\textbf{\subsection*{UKŁAD DOŚWIADCZALNY}}
\end{center}


\begin{center}
\textbf{\subsection*{WYNIKI POMIARÓW}}
\end{center}


\begin{center}
\textbf{\subsection*{ANALIZA DANYCH}}
\end{center}



\begin{center}
\textbf{\subsection*{DYSKUSJA WYNIKÓW I WNIOSKI}}
\end{center} 


\begin{center}
\begin{thebibliography}{9}


\end{thebibliography}

\end{center}


\end{document}