\documentclass[10pt,a4paper]{article}
\usepackage[utf8]{inputenc}
\usepackage{amsmath}
\usepackage{gensymb}
\usepackage{amsfonts}
\usepackage{siunitx}
\usepackage[european]{circuitikz}
\usepackage{geometry}
\newgeometry{tmargin=2cm, bmargin=2cm, lmargin=2cm, rmargin=2cm}
\usepackage{amssymb}
\usepackage{multirow}
\usepackage{polski}
\usepackage{graphicx}
\author{\textbf{T. Fąs}}
\title{\textbf{WYZNACZANIE PRZERWY ENERGETYCZNEJ InSb}}
\begin{document}
\maketitle

\begin{center}
\textbf{\subsection*{STRESZCZENIE}}
\end{center}
W doświadczeniu wyznaczono prezerwę energetyczną $E_{G}$ półprzewodnika InSb. Otrzymano wartości: $E_{G}=0,2819\pm0,0019$ eV. Wartość ta jest zgodna z wartością rzeczywistą, która wynosi $0,2511 \pm 0,095$ eV.


\begin{center}
\textbf{\subsection*{WSTĘP}}
\end{center}

W przypadku półprzewodników, takich jak InSb, w paśmie przewodnictwa jest obecnych niewiele wolnych elektronów, przez co prąd praktycznie nie może płynąć. Spowodowanie jest to istnieniem przerwy energetycznej $E_{G}$ między pasmem przewodnictwa, a pasmem walencyjnym. Gdy dostarczy się elektronom dodatkową energię, pozwalającą na pokonanie przerwy energetycznej, to półprzewodnik będzie zdolny do przewodzenia prądu.

Półprzewodniki charakteryzują się też dużą podatnością na zmiany temperatury. W przypadku półprzewodników samoistnych zależność oporu $R$ od temperatury $T$ wyraża się wzorem:

\begin{equation}
R\left(T\right)=R_{0}\exp{\left(\dfrac{E_{G}}{2k_{B}T}\right)},
\end{equation}
gdzie $k_{B}$ jest stałą Boltzmanna. Tak więc mierząc zależność oporu od temperatury można poznać wartość przerwy energetycznej. 


\begin{center}
\textbf{\subsection*{UKŁAD DOŚWIADCZALNY}}
\end{center}

W pomiarach wykorzystano: miernik CHY 38 do pomiaru oporu próbki, woltomierz mierzący napięcie z dokładnością do 0,001 mV, próbkę InSb przytwierdzoną do podstawki miedzianej oraz ciekły azot do chłodnienia próbki. Do półprzewodnika była podłączona termopara miedź-konstantan, której drugi koniec był zanurzony w naczyniu z ciekłym azotem. Przy pomocy woltomierza mierzono napięcie na termoparze, które następnie przeliczano na temperaturę półprzewodnika. Próbka była stopniowo zanurzana w naczyniu z ciekłym azotem, co pozwalało na jej równomierne schładzanie. Wskazania woltomierza i amperomierza były nagrywane, by później móc wybrać najbardziej wiarygodne wyniki, czyli takie, dla których próbka osiągała stan stacjonarny.

\begin{center}
\textbf{\subsection*{WYNIKI POMIARÓW}}
\end{center}
W Tabeli 1 przedstawiono wyniki pomiarów dla wybranych punktów.

\begin{table}[h!]
\centering
\caption{Wyniki pomiarów.}
\begin{tabular}{|c|c|c|c|c|c|c|c|}
\hline
$U$ {[}mV{]} & $T$ {[}K{]} & $R$ {[}$\Omega${]} & $u_{R}$ {[}$\Omega${]} & $U$ {[}mV{]} & $T$ {[}K{]} & $R$ {[}$\Omega${]} & $u_{R}$ {[}$\Omega${]} \\ \hline
6,359        & 295,9558    & 32,5               & 0,56                   & 2,975        & 201,5979    & 392                & 6,136                  \\ \hline
6,099        & 289,3208    & 37                 & 0,596                  & 2,965        & 201,2822    & 394                & 6,152                  \\ \hline
5,889        & 283,9077    & 41,3               & 0,6304                 & 2,911        & 199,5716    & 416                & 6,328                  \\ \hline
5,728        & 279,7231    & 45,1               & 0,6608                 & 2,548        & 187,8057    & 565                & 7,52                   \\ \hline
5,596        & 276,2687    & 48,8               & 0,6904                 & 2,524        & 187,0101    & 576                & 7,608                  \\ \hline
5,409        & 271,337     & 54,4               & 0,7352                 & 2,404        & 182,9965    & 626                & 8,008                  \\ \hline
5,324        & 269,0801    & 57,3               & 0,7584                 & 2,307        & 179,7066    & 667                & 8,336                  \\ \hline
5,005        & 260,5197    & 70,3               & 0,8624                 & 2,142        & 174,0097    & 725                & 8,8                    \\ \hline
4,974        & 259,6799    & 71,7               & 0,8736                 & 1,96         & 167,5653    & 784                & 9,272                  \\ \hline
4,948        & 258,9744    & 73,6               & 0,8888                 & 1,782        & 161,081     & 832                & 9,656                  \\ \hline
4,364        & 242,8372    & 112,9              & 1,2032                 & 1,627        & 155,2705    & 868                & 9,944                  \\ \hline
4,301        & 241,0606    & 118,6              & 1,2488                 & 1,611        & 154,6613    & 871                & 9,968                  \\ \hline
3,699        & 223,677     & 194                & 4,552                  & 1,326        & 143,4783    & 923                & 10,384                 \\ \hline
3,686        & 223,2927    & 197                & 4,576                  & 1,238        & 139,8846    & 937                & 10,496                 \\ \hline
3,681        & 223,1448    & 198                & 4,584                  & 0,883        & 124,552     & 992                & 10,936                 \\ \hline
3,678        & 223,0561    & 199                & 4,592                  & 0,853        & 123,1848    & 995                & 10,96                  \\ \hline
3,278        & 211,0166    & 293                & 5,344                  & 0,785        & 120,0376    & 1006               & 11,048                 \\ \hline
\end{tabular}
\end{table}

\begin{center}
\textbf{\subsection*{ANALIZA DANYCH}}
\end{center}
 Napięcie $U$ na termoparze  przeliczono na temperaturę $T$ korzystając z zależności: $T=100\sqrt{0.034U^2+1.07U+0.58}$.  Niepewność $u_{R}$ oporu obliczono, korzystając z instrukcji miernika. Wyniki tych obliczeń są przedstawione w Tabeli 1, wraz z odpowiadającymi im wartościami napięcia i oporu.
 
 
 Wartości temperatury i oporu z Tabeli 1 naniesiono na wykres przedstawiony na Rysunku 1.
 
 \begin{figure}[h!]
\centering
\begin{minipage}{0.5\textwidth}
  \centering
  \includegraphics[width=8cm, height=6cm ]{rap16rys1} 
\caption{Zależność oporu od temperatury.}
\end{minipage}%
\begin{minipage}{0.5\textwidth}
  \centering
  \includegraphics[width=8cm, height=6cm ]{rap16rys2} 
\caption{Krzywa najlepszego dopasowania.}
\end{minipage}
\end{figure}

Jak widać wykres ten nie podlega zależności z Równania (1). Wynika do z faktu, iż Równanie (1) jest prawdziwe tylko dla czystej próbki, tymczasem rzeczywiste półprzewodnik są zanieczyszczone różnymi innymi związkami, których udział dominuje w niskich temperaturach.   Dlatego też należy ograniczyć obszar dopasowywania krzywej do wyników pomiarów do zakresu 205-300 K. W takim zakresie temperatur właściwości InSb są dominujące i punkty podlegają zależności z Równania (1).

Do dopasowania krzywej wykorzystano program \textit{gnuplot}, zakres danych obejmował 17 punktów pomiarowych, a dopasowywania krzywa miała postać Równania (1). Otrzymano wartości $R_{0}=0,1314\pm0,0057$ $\Omega$, $E_{G}/2k_{B}=1636\pm11$ K, a samą krzywą przedstawiono na Rysunku 2. Wartość $\chi^2$ wynosi 17,45 i jest to wartość niższa od wartości krytycznej, która dla 15 stopni swobody i wartości p=0,05 wynosi 27,59. Tak więc dane z przedziału (205-300)K nie przeczą Równaniu (1). Z wartości $E_{G}/2k_{B}=1636$ K wyznaczono $E_{G}=0,2819$ eV kładąc $k_{B}=8,617\cdot10^{-5}$ eV/K. Niepewność tej wielkości wyznaczono, korzystając z metody propagacji małych błędów.
Wzór przenoszenia niepewności w tej metodzie jest następujący:

 \begin{equation}
 u_{f}^2=\sum_{i=1}^n \left( \dfrac{\partial f}{\partial x_{i}}u_{i}\right)^2
 \end{equation}
 gdzie wielkość $f$ zależy od wielkości $x_{i}$ o niepewnościach $u_{i}$ \cite{tay1}. Przy założeniu, że stałą Boltzmanna znamy dokładnie, wartość niepewności przerwy energetycznej wynosi 0,0019 eV. Ostatecznie $E_{G}=0,2819\pm0,019$ eV. Wartość rzeczywista wynosi $0,2511\pm0,095$ eV \cite{eg}. Różnica tych wielkości wynosi 0,031 eV, a niepewność tej różnicy, obliczona z Równania (2) wynosi 0,095 eV. Tak więc różnica jest mniejsza od trzykrotności jej niepewności, więc na mocy testu $3\sigma$ otrzymana wartość $E_{G}$ jest zgodna z wartością rzeczywistą.

\begin{center}
\textbf{\subsection*{DYSKUSJA WYNIKÓW I WNIOSKI}}
\end{center} 

Dysponując podstawowymi narzędziami, takimi jak woltomierz, omomierz oraz ciekły azot, udało się uzyskać wartości zgodne z wartościami rzeczywistymi oraz cechujące się niższą niepewnością. Nie wykryto również żadnych anomalii, otrzymane dane były zgodne z przewidywaniami. Warto brać pod uwagę to, iż ciężko było liczyć na uzyskanie stanu stacjonarnego w doświadczeniu i wszystkie zebrane dane nie są tak naprawdę wartościami, które dokładnie sobie odpowiadają. Lecz mimo to udało się uzyskać wysoką zgodność wyników z przewidywaniami.Cały eksperyment można uznać za zakończony sukcesem.
\begin{center}
\begin{thebibliography}{9}

 \bibitem{tay1}
 J. R. Taylor,
 \emph{Wstęp do analizy błędu pomiarowego},
 PWN, Warszawa, 1995, s. 175.

\bibitem{eg}
H. Bernot, H. Hinsch
 \emph{Determination of the intrinsic carrier concentration in InSb by helicon waves},
 Applied Physics, Volume 1, Issue 3, s.147-151
 
 \end{thebibliography}

\end{center}


\end{document}