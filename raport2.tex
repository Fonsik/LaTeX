\documentclass[10pt,a4paper]{article}
\usepackage[utf8]{inputenc}
\usepackage{amsmath}
\usepackage{amsfonts}
\usepackage{geometry}
\newgeometry{tmargin=2cm, bmargin=2cm, lmargin=2cm, rmargin=2cm}
\usepackage{amssymb}
\usepackage{polski}
\usepackage{graphicx}
\author{\textbf{T. Fąs}}
\title{\textbf{WYZNACZANIE GĘSTOŚCI WALCA}}
\begin{document}
\maketitle

\begin{center}
\textbf{\subsection*{STRESZCZENIE}}
\end{center}
Celem doświadczenia było wyznaczenie gęstości ciała w kształcie walca. Wykorzystano trzy różne metody wyznaczania gęstości, mierząc masę i objętość ciała. Otrzymano trzy wyniki: $\rho_{A}=(7,8462\pm0,0043)$ g/cm$^3$, $\rho_{B}=(7,85\pm0,36)$ g/cm$^{3}$ i $\rho_{C}=(7,8564\pm 0,0071)$ g/cm$^3$. 
\begin{center}
\textbf{\subsection*{WSTĘP}}
\end{center}
Dla ciała o masie $m$ i o jednorodnym rozkładzie masy gęstość $\rho$ jest dana wzorem:
\begin{equation}
\rho=\dfrac{m}{V},
\end{equation}
gdzie $V$ jest objętością ciała \cite{hrw1}. Celem doświadczenia było wyznaczenie gęstości ciała przy pomocy trzech różnych metod. Objętość ciała wyznaczono dwoma różnymi sposobami, z kolei trzecia metoda opierała się na wykorzystaniu prawa Archimedesa. Masę ciała wyznaczono jednym sposobem, przy pomocy wagi, i wykorzystano ją w każdej analizie danych. Pierwsza z metod polegała na wyznaczeniu objętości poprzez pomiar jego wysokości $H$ i średnicy $D$. Objętość walca dana jest wzorem:
\begin{equation}
V=\pi R^{2}H=\dfrac{\pi D^{2}H}{4},
\end{equation}
gdzie $R$ jest promieniem walca, $D=2R$. Podstawiając Równanie (2) do Równania (1) otrzymano:
\begin{equation}
\rho_{A}=\dfrac{4m}{\pi d^{2}H}
\end{equation}
Druga metoda polegała na wyznaczeniu objętości ciała z różnicy poziomu cieczy w menzurce. Odczytywano poziom cieczy $V_{1}$ w menzurce, następnie całkowicie zanurzano w niej walec i odczytywano nowy poziom $V_{2}$. Z różnicy poziomów znajdowano objętość ciała. Szukana objętość dana jest wzorem:
\begin{equation}
V=V_{2}-V_{1}
\end{equation}
Po podstawieniu Równania (4) do Równania (1) otrzymano:
\begin{equation}
\rho_{B}=\dfrac{m}{V_{2}-V_{1}}
\end{equation}
Trzecia metoda wykorzystywała prawo Archimedesa. Rozważmy ciało zanurzone całkowicie w zlewce. Ciało to nie dotyka dna zlewki. W związku z tym siła wyporu działająca na ciało jest równa ciężarowi ciała. Z prawa Archimedesa wiadomo, że ciężar ciała jest równy ciężarowi wypartej cieczy, czyli: 
\begin{equation}
F_{wyp.}=\rho gV;
\end{equation}
gdzie $g$ jest przyśpieszeniem ziemskim, $\rho$ jest gęstością cieczy, a $V$ objętością wypartej cieczy \cite{hrw2}. Jeśli wyznaczono masę zlewki z wodą przed wprowadzeniem ciała do zlewki i po, to korzystając z Równania (6) i Równania (1) można wyznaczyć gęstość ciała:
\begin{equation}
\rho_{C}=\dfrac{\rho_{w} m}{m_{zwp}-m_{zw}},
\end{equation}
gdzie $\rho_{w}$ jest gęstością wody, $m_{zwp}$ jest masą zlewki z wodą i ciałem, a $m_{zw}$ masą zlewki z wodą.
\begin{center}
\textbf{\subsection*{UKŁAD DOŚWIADCZALNY}}
\end{center}
Ciało, którego gęstość wyznaczano, miało kształt walca. Z góry założono, że jest to walec o jednorodnym rozkładzie masy. Do jednego z denek walca była przyczepiona cienka nić. Pomiaru masy $m$ walca dokonano przy pomocy wypoziomowanej i wytarowanej wagi pozwalającej na pomiar z dokładnością $\Delta_{m}=0,01$ g. 
Do pomiarów tych wielkości skorzystano z suwmiarki o dokładności $\Delta_{s}=0,01$ mm. Wysokość, jak i średnicę, zmierzono w różnych miejscach walca. Daje to pewność co do tego, czy obiekt badany jest rzeczywiście walcem. W drugiej metodzie wykorzystano menzurkę z wodą oraz statyw. Działka odczytu menzurki wynosiła $\Delta_{V}=1$ cm$^{3}$.
Najpierw odczytywano poziom wody w menzurce, który oznaczono jako $V_{1}$, a następnie wprowadzano do menzurki walec zawieszony na statywie tak, aby był on całkowicie zanurzony. Odczytywano nowy poziom wody, oznaczony jako $V_{2}$ i wyznaczano ich różnicę. W niektórych sytuacjach, gdy poziom wody nie pokrywał się z działką odczytu za wynik przyjmowano najbliższą działkę odczytu. W trzeciej metodzie wykorzystano wagę, zlewkę z wodą destylowaną oraz termometr. W pierwszej kolejności mierzono temperaturę wodzy oraz ważono zlewkę z wodą. Masę tego układu oznaczono jako $m_{zw}$. Następnie wprowadzano do zlewki walec zawieszony na statywie tak, aby był on całkowicie zanurzony, ale nie dotykał dna i odczytywano nową masę układu, oznaczoną przez $m_{zwp}$. Pomiędzy kolejnymi pomiarami wykonywanymi metodami drugą i trzecią oczyszczano walec z wody.
\begin{center}
\textbf{\subsection*{WYNIKI POMIARÓW}}
\end{center}
Masę walca zmierzono pięciokrotnie, a jego wymiary dziesięciokrotnie. W pozostałych metodach wykonano pięć serii pomiarowych.
W Tabeli 1 przedstawiono wyniki pomiarów masy $m$ walca, objętości $V_{1}$ i $V_{2}$ wyznaczanych menzurką, ich różnicę $\Delta V$ oraz wyniki pomiarów $m_{zw}$ i $m_{zwp}$. 
\begin{center}
 \begin{table}[h!]
 \label{1}
 \centering
 \caption{Wyniki pomiarów }
 \begin{tabular}{|c|c|c|c|c|c|}
 \hline
 $m$ [g] &$V_{1}$ [cm$^{3}$]  & $V_{2}$ [cm$^{3}$]& $\Delta V$ [cm$^{3}$] & $m_{zw}$ [g] & $m_{zwp}$ [g] \\
 \hline
 141,21& 48 &62 &14 &333,40 &351,35 \\
 \hline
141,22 & 49 & 67 & 18 & 333,13&351,07 \\
 \hline
 141,20 & 48 & 61 & 13 & 332,56&350,55 \\
 \hline
141,22&50&68&18&332,12&350,04\\
 \hline
 141,21&53&71&18&331,93&349,84\\
 \hline
 \end{tabular}
 \end{table}
 \end{center}
 W Tabeli 2 Przedstawiono wyniki pomiarów średnicy $D$ i wysokości $H$ walca.
 \begin{center}
 \begin{table}[h!]
 \label{2}
 \centering
 \caption{Wyniki pomiarów średnicy $D$ i wysokości $H$ walca }
 \begin{tabular}{|c|c|c|c|c|c|c|c|c|c|c|}
 \hline
 $D$ [mm] &23,96& 23,95& 23,97 & 23,95& 23,95&23,96&23,95&23,95&23,96&23,95 \\
 \hline
$H$ [mm]& 39,91 &39,93&39,92&39,93&39,92&39,95&39,96&39,93&39,95&39,93 \\
\hline

 \end{tabular}
 \end{table}
 \end{center}
 Dodatkowo, przy wykonywaniu pomiarów metodą numer 3 zmierzono temperaturę, która wynosiła $20,6$ $^{\circ}$C. Wartość gęstości wody dla tej temperatury wynosi $\rho_{w}=0,998$ g/cm$^3$. Dodatkowo założono, że wielkość nie jest obarczona błędem pomiarowym \cite{wod}.
 \begin{center}
\textbf{\subsection*{ANALIZA DANYCH}}
\end{center}
 Niepewności pomiarowe wielkości mierzonych bezpośrednio wyznaczono korzystając z zależności:
 \begin{equation}
 u^{2}=s_{\bar{x}}^{2}+\dfrac{\Delta^{2}}{3},
 \end{equation}
 gdzie $s_{\bar{x}}$ jest odchyleniem standardowym średniej wielkości $x$, a $\Delta$ jest dopuszczalnym, maksymalnym błędem pomiaru \cite{tay1}. Korzystając z rozwinięcia $s_{\bar{x}}$, Równanie (8) można zapisać jako:
  \begin{equation}
 u^{2}= \dfrac{1}{N(N-1)} \sum_{i=1}^{N} (x_{i}- \bar{x})^{2}+\dfrac{\Delta^{2}}{3},
 \end{equation}
 gdzie $N$ jest liczbą wykonanych pomiarów danej wielkości, a $\bar{x}$ jest średnią danej wielkości.
 Aby uczynić analizę bardziej kompletną obliczono też wartość niepewności pojedynczego pomiaru, która dana jest wzorem:
   \begin{equation}
 s_{x}^{2}= \dfrac{1}{N-1} \sum_{i=1}^{N} (x_{i}- \bar{x})^{2} \quad \cite{tay2} .
 \end{equation}
 Na podstawie Równania (9) i Równania (10) stworzono Tabelę 3, w której znajdują się średnie mierzonych wielkości, ich niepewności pojedynczego pomiaru i niepewności średniej oraz maksymalne błędy pomiaru.
  
  \begin{center}
 \begin{table}[h!]
 \label{3}
 \centering
 \caption{Tabela niepewności pomiarów }
 \begin{tabular}{|c|c|c|c|}
 \hline
 Wielkość &$m$ [g]& $D$ [cm]& $H$ [cm]  \\
 \hline
Średnia & 141,2120 &2,39550&3,99330 \\
\hline
Niepewność pojedynczego pomiaru&0,0084&0,00071&0,00157\\
\hline
Niepewność średniej&0,0037&0,00022&0,00050\\
\hline
Dopuszczalny błąd pomiaru $\Delta$& 0,01&0,001&0,001\\
\hline
$\Delta / \sqrt{3}$&0,0058&0,00058&0,00058\\
\hline
$u$&0,0069&0,00062&0,00076\\
\hline
 \end{tabular}
 \end{table}
 \end{center}
 Korzystając z wartości średnich znajdujących się w Tabeli 3 i wartości $\pi=3,141592653$ wyliczono gęstość ciała z Równania (3), $\rho_{A}=7,8462$ g/cm$^3$. Jako, że niepewność tego wyniku jest niepewnością złożoną to skorzystano z metody propagacji małych błędów, która pozwala na przenoszenie niepewności pomiarowych. Jeśli szukana wielkość wielkość jest funkcją zależną od innych wielkości fizycznych, to jej niepewność dana jest wzorem:
 \begin{equation}
 u_{f}^2=\sum_{i=1}^n \left( \dfrac{\partial f}{\partial x_{i}}u_{i}\right) ^2,
 \end{equation}
 gdzie $u_{f}$ jest niepewnością szukanej wielkości zależnej od $x_{i}$, a $u_{i}$ jest niepewnością $x_{i}$ \cite{tay3}.
 Stosując Równanie (11) do Równania (3) otrzymano:
 \begin{equation}
 u_{\rho_{A}}^2=\rho_{A}^2 \left[ \left(\dfrac{u_{m}}{\bar{m}}\right)^2+\left(\dfrac{u_{H}}{\bar{H}}\right)^2+\left(\dfrac{2u_{D}}{\bar{D}}\right)^2 \right ],
 \end{equation}
 gdzie $u_{m}$, $u_{H}$, $u_{D}$ są kolejno niepewnościami masy ciała, jego wysokości i średnicy. Obliczona wartość $u_{\rho_{A}}$ wynosi $0,0043$ g/cm$^3$.
 Ostateczny wynik można zapisać jako: $\rho_{A}=(7,8462\pm0,0043)$ g/cm$^3$.
 Tabela 4 jest analogiczna do Tabeli 3, jednak tym razem wyznaczono niepewności dla objętości mierzonych przy pomocy menzurki. 
 
  \begin{center}
 \begin{table}[h!]
 \label{4}
 \centering
 \caption{Tabela niepewności pomiarów }
 \begin{tabular}{|c|c|c|}
 \hline
 Wielkość &$V_{1}$ [cm$^{3}$] & $V_{2}$ [cm$^{3}$]  \\
 \hline
Średnia & 49,6 &65,8 \\
\hline
Niepewność pojedynczego pomiaru&2,07364&4,20714\\
\hline
Niepewność średniej&0,92736&1,88149\\
\hline
Dopuszczalny błąd pomiaru $\Delta$& 1&1\\
\hline
$\Delta / \sqrt{3}$&0,57735&0,57735\\
\hline
$u$&0,57735&0,57735\\
\hline
 \end{tabular}
 \end{table}
 \end{center}
 Podstawiając wartości średnie do Równania (5) otrzymano wynik: $\rho_{B}=8,72$ g/cm$^{3}$. Niepewność tego wyniku otrzymano podstawiając Równanie (5) do Równania (11) jako funkcję f. Otrzymano następujący wzór:
 \begin{equation}
  u_{\rho_{B}}^2=\rho_{B}^2 \left[ \left(\dfrac{u_{m}}{\bar{m}}\right)^2+\left(\dfrac{u_{\bar{V}}}{\bar{V}}\right)^2\right ],
 \end{equation}
 gdzie $\bar{V}$ jest średnią różnicą $V_{1}$ i $V_{2}$, a $u_{\bar{V}}$ jest jej niepewnością daną wzorem:
 \begin{equation}
 u_{\bar{V}}^2=u_{V_{1}}^2+u_{V_{2}}^2+s_{V}^2,
 \end{equation}
 gdzie $u_{V_{1}}$ i $u_{V_{1}}$ są niepewnościami $V_{1}$ i $V_{2}$, w tym przypadku są to niepewności pomiarowe przyrządu, z kolei $s_{V}$ jest niepewnością statystyczną średniej różnicy, która jest dana wzorem:
 \begin{equation*}
  s_{V}^2=s_{V_{1}}^2+s_{V_{2}}^2-2 \sum_{i=1}^{5} (V_{1_{i}}- \bar{V}_{1})(V_{2_{i}}- \bar{V}_{2}).
 \end{equation*}
  Wzór (14) wyznaczono metodą propagacji małych błędów.
  Na tej podstawie obliczono wartość $u_{\rho_{B}}=0,74$ g/cm$^{3}$. Ostateczny wynik zapisano jako: $\rho_{B}=(8,72\pm0,74)$ g/cm$^{3}$. Jak widać, wynik ten jest obarczony dużą niepewnością w porównaniu do metody A, jak i odbiega on znacznie od wartości $\rho_{A}$. Analiza wyników pomiarów zapisanych w Tabeli 1 ujawnia źródło tej rozbieżności: dwa z pięciu pomiarów dają znacznie mniejszy wkład do średniej objętości próbki. w związku z tym postanowiono odrzucić te pomiary i wykorzystać tylko te, których $\Delta V=18$ cm$^3$. Na ich podstawie stworzono Tabelę 5.
 
 \begin{center}
 \begin{table}[h!]
 \label{5}
 \centering
 \caption{Tabela niepewności pomiarów }
 \begin{tabular}{|c|c|c|}
 \hline
 Wielkość &$V_{1}$ [cm$^{3}$] & $V_{2}$ [cm$^{3}$]  \\
 \hline
Średnia & 50,6667 &68,6667 \\
\hline
Niepewność pojedynczego pomiaru&2,08167&2,08167\\
\hline
Niepewność średniej&1,20185&1,20185\\
\hline
Dopuszczalny błąd pomiaru $\Delta$& 1&1\\
\hline
$\Delta / \sqrt{3}$&0,57735&0,57735\\
\hline
$u$&0,57735&0,57735\\
\hline
 \end{tabular}
 \end{table}
 \end{center}
Po podstawieniu tych wartości do Równania (5) i Równania (13) oraz Równania (14) otrzymano ostateczny wynik: $\rho_{B}=(7,85\pm0,36)$ g/cm$^{3}$. Otrzymany wynik pokrywa się z $\rho_{A}$, a jego niepewność jest znacznie mniejsza. 

Tabela 6 przedstawia niepewności pomiaru masy zlewki z wodą, $m_{zw}$, oraz zlewki z wodą i zanurzonym walcem, $m_{zwp}$.
 
  \begin{center}
 \begin{table}[h!]
 \label{6}
 \centering
 \caption{Tabela niepewności pomiarów }
 \begin{tabular}{|c|c|c|c|}
 \hline
 Wielkość &$m_{zw}$ [g] & $m_{zwp}$ [g]  \\
 \hline
Średnia & 332,63 &350,57 \\
\hline
Niepewność pojedynczego pomiaru&0,63204&0,64665\\
\hline
Niepewność średniej&0,28266&0,28919\\
\hline
Dopuszczalny błąd pomiaru $\Delta$& 0,01&0,01\\
\hline
$\Delta / \sqrt{3}$&0,00577&0,00577\\
\hline
$u$&0,28271&0,28925\\
\hline
 \end{tabular}
 \end{table}
 \end{center}
 Gęstość $\rho_{C}$ obliczono z Równania (7), gdzie podstawiono średnie wartości $m_{zw}$ i $m_{zwp}$ oznaczone jako $\bar{m}_{zw}$ i $\bar{m}_{zwp}$. Za gęstość wody, $\rho_{w}$ podstawiono wartość 0,99821 g/cm$^3$. Dodatkowo gęstość wody potraktowano jako wartość nieobarczoną błędem pomiarowym. Otrzymano wartość $\rho_{C}=7,85638$ g/cm$^3$. Niepewność tego pomiaru wyznaczono korzystając z Równania (7) i Równania (11). Otrzymano wzór:
 \begin{equation}
 u_{\rho_{C}}^2=\rho_{C}^2 \left[ \left(\dfrac{u_{m}}{\bar{m}}\right)^2+\left(\dfrac{u_{m_{ww}}}{\bar{m}_{ww}}\right)^2 \right ],
 \end{equation}
 gdzie $\bar{m}_{ww}$ jest średnią różnicą $m_{zw}$ i $m_{zwp}$, a $u_{m_{ww}}$ jest jej niepewnością daną wzorem:
 \begin{equation}
  u_{m_{ww}}^2=s_{mww}^2+\dfrac{2\Delta_{m}^2}{3},
  \end{equation} 
  gdzie $\Delta_{m}$ jest dopuszczalnym błędem granicznym wagi, a $s_{mww}$ jest odchyleniem standardowym średniej różnicy masy, które jest dane wzorem:
  \begin{equation}
 s_{mww}^2=s_{m_{zw}}^2+s_{m_{zwp}}^2-2 \sum_{i=1}^{5} (m_{zw_{i}}- \bar{m}_{zw})(m_{zwp_{i}}- \bar{m}_{zwp}).
 \end{equation}
  Podstawiając dane z Tabeli 6 do Równania (15) otrzymano $u_{\rho_{C}}=0,0071$ g/cm$^3$. Ostateczny wynik to $\rho_{C}=(7,8564\pm 0,0071)$ g/cm$^3$. W Tabeli 7 przedstawiono wszystkie otrzymane gęstości wraz z ich niepewnościami oraz procentową niepewność wyniku, to znaczy, stosunek niepewności do końcowego wyniku. Ilość cyfr znaczących wybrano na podstawie najdokładniejszego wyniku.
  
   \begin{center}
 \begin{table}[h!]
 \label{7}
 \centering
 \caption{Tabela otrzymanych gęstości}
 \begin{tabular}{|c|c|c|c|}
 \hline
 Metoda &A & B&C  \\
 \hline
Gęstość [g/cm$^3$] & 7,8462 &7,8451&7,8564 \\
\hline
Niepewność [g/cm$^3$] &0,0043&0,3559&0,0071\\
\hline
Niepewność względna [\%] &0,055&4,54&0,090\\
\hline
 \end{tabular}
 \end{table}
 \end{center}
 \begin{center}
\textbf{\subsection*{DYSKUSJA WYNIKÓW I WNIOSKI}}
\end{center}
Wyniki zebrane w Tabeli 7 pozwalają sądzić, iż walec był wykonany ze stali nierdzewnej. Gęstość stali nierdzewnej wynosi $\rho_{s}=7,8$ g/cm$^3$ w $20$ $^{\circ}$C \cite{st}. Widać wyraźnie wysoką zgodność wyników oraz ich precyzję. Niezwykle precyzyjna okazała się metoda A, której niepewność względna wynosi $0,0043 \%$. Najmniej dokładna okazała się metoda B, z dokładnością $4,54\%$. Wynika to z niskiej rozdzielczości działki menzurki, która byłą tylko o rząd wielkości mniejsza od mierzonej wielkości. W przypadku metody A działka przyrządu była aż o 4 rzędy wielkości mniejsza od mierzonej wielkości. W przypadku metody B był to znacznie większy wkład niż w przypadku metody A. Dodatkowo w przypadku metody B na wynik pomiaru wpływał też kąt patrzenia obserwatora oraz powstawanie menisku. Gdyby udało się wyeliminować zjawisko menisku, powstawanie pęcherzyków powietrza jak i zastosować działkę o większej rozdzielczości, to metoda wykorzystująca menzurkę byłaby dokładniejsza. Bardzo wysoką dokładność posiada również metoda C, o niepewności względnej $0,09 \%$. Dodatkowo metoda ta jest uniwersalna, ponieważ nie zależy od kształtu badanego ciała. Pomimo tych wszystkich niepewności otrzymane wyniki są bardzo bliskie sobie oraz bliskie gęstości stali podanej w tablicach.

\begin{center}
\begin{thebibliography}{9}

\bibitem{hrw1}
 D. Halliday, R. Resnick, J. Walker,
  \emph{Podstawy Fizyki, Tom 2},
  PWN, Warszawa, 2003, s. 62.
 
 \bibitem{hrw2}
 D. Halliday, R. Resnick, J. Walker,
  \emph{Podstawy Fizyki, Tom 2},
  PWN, Warszawa, 2003, s. 73.
 
  \bibitem{wod}
 Praca zbiorowa,
  \emph{Tablice fizyczno-astronomiczne},
  Wydawnictwo Adamantan, Warszawa, 2002
  
  \bibitem{tay1}
 J. R. Taylor,
 \emph{Wstęp do analizy błędu pomiarowego},
 PWN, Warszawa, 1995, s. 71.
 
 \bibitem{tay2}
 J. R. Taylor,
 \emph{Wstęp do analizy błędu pomiarowego},
 PWN, Warszawa, 1995, s. 101.
 
 \bibitem{tay3}
 J. R. Taylor,
 \emph{Wstęp do analizy błędu pomiarowego},
 PWN, Warszawa, 1995, s. 87.
 
  \bibitem{st}
 Praca zbiorowa,
 \emph{Stainless Steel: Tables of Technical Properties},
 Euro Inox, Luxemburg, 2007, s. 18.

\end{thebibliography}
\end{center}
 
\end{document}