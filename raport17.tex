\documentclass[10pt,a4paper]{article}
\usepackage[utf8]{inputenc}
\usepackage{amsmath}
\usepackage{gensymb}
\usepackage{amsfonts}
\usepackage{siunitx}
\usepackage[european]{circuitikz}
\usepackage{geometry}
\newgeometry{tmargin=2cm, bmargin=2cm, lmargin=2cm, rmargin=2cm}
\usepackage{amssymb}
\usepackage{multirow}
\usepackage{polski}
\usepackage{graphicx}
\author{\textbf{T. Fąs}}
\title{\textbf{DOŚWIADCZENIE YOUNGA PRZY WYKORZYSTANIU MIKROFAL}}
\begin{document}
\maketitle

\begin{center}
\textbf{\subsection*{STRESZCZENIE}}
\end{center}
W doświadczeniu próbowano powtórzyć doświadczenie z dwoma szczelinami przy wykorzystaniu mikrofal. Jednakże otrzymany wzór interferencyjny nie jest zgodny z przewidywaniami teoretycznymi.

\begin{center}
\textbf{\subsection*{WSTĘP}}
\end{center}

Jeśli fala płaska przejdzie przez szczelinę, to odchodzi do zjawiska dyfrakcji, czyli ugięcia fali na krawędziach szczeliny. Rozkład natężenia $I$ fali na ekranie równoległym do szczeliny i prostopadłym do biegu wiązki jest dany wzorem:

\begin{equation}
 I=I_{0}\left(\dfrac{sin\left(\pi l \sin\left(\theta\right)/\lambda\right)}{\pi l \sin\left(\theta\right)/\lambda}\right)^2,
 \end{equation} 

gdzie $\lambda$ to długość fali, $\l$ to szerokość szczeliny, a $\theta$ to kąt zawarty między prostą łączącą punkt na ekranie i środek szczeliny, a między prostą prostopadłą do ekranu i przechodzącą przez środek szczeliny.

Jeżeli mamy do czynienia z dwoma szczelinami, to dochodzi dodatkowo do zjawiska interferencji: amplituda fali jest sumą amplitud przechodzących przez obie szczeliny. 
Z tego powodu Równanie (1) ulega modyfikacji i otrzymujemy:

\begin{equation}
I=2I_{0}\left[1+\gamma \cos\left(2\pi d\sin(\theta/\lambda)\right)\right]\left(\dfrac{\sin\left(\pi l \sin(\theta)/\lambda\right)}{\pi l \sin(\theta)/\lambda}\right)^2,
\end{equation}
gdzie $\gamma$ jest współczynnikiem spójności wiązek, a $d$ jest odległością między środkami szczelin. 

W doświadczeniu, zamiast natężenia, mierzono napięcie na detektorze, które było wprost proporcjonalne do natężenia fali. Celem doświadczenia było zbadanie, czy dane eksperymentalne zgadzają się z Równaniem (2).

\begin{center}
\textbf{\subsection*{UKŁAD DOŚWIADCZALNY}}
\end{center}
Układ doświadczalny składał się z generatora mikrofal, detektora, miernika napięcia oraz ze ścianki, która pozwalała na kontrolowanie liczby szczelin, ich szerokości i wzajemnej odległości między nimi. Detektor znajdował się na obrotowym ramieniu, które w łatwy sposób pozwalało na zmierzenie kąta $\theta$. Samo ramię było zaczepione do środka ścianki. 


\begin{center}
\textbf{\subsection*{WYNIKI POMIARÓW}}
\end{center}
Wartości napięć i kątów dla pojedynczej szczeliny zamieszczono w Tabelach 1-4, wyniki pomiarów dla dwóch szczelin są przedstawione w Tabelach 5 i 6.

\begin{table}[h!]
\centering
\caption{Wyniki pomiarów dla $l=0,1$ m.}
\begin{tabular}{|c|c|c|c|c|c|c|c|c|c|}
\hline
$\theta$ [$^\circ$] & 0   & 5   & 10 & 15  & 20  & -5  & -10 & -15 & -20 \\ \hline
I [mV]              & 2,5 & 2,5 & 2  & 0,9 & 0,1 & 2,1 & 1,1 & 0,5 & 0,5 \\ \hline
\end{tabular}
\end{table}

\begin{table}[h!]
\centering
\caption{Wyniki pomiarów dla $l=0,05$ m.}
\begin{tabular}{|c|c|c|c|c|c|c|c|c|c|c|c|c|}
\hline
$\theta$ [$^\circ$] & 0   & 5   & 10  & 15   & 20   & -5  & -10  & -15  & -20 & 13   & 12   & -25   \\ \hline
I [mV]              & 0,6 & 0,7 & 0,8 & 0,75 & 0,35 & 0,7 & 0,65 & 0,65 & 0,3 & 0,75 & 0,75 & 0,175 \\ \hline
\end{tabular}
\end{table}


\begin{table}[h!]
\centering
\caption{Wyniki pomiarów dla $l=0,07$ m.}
\begin{tabular}{|c|c|c|c|c|c|c|c|c|c|c|c|c|c|}
\hline
$\theta$ [$^\circ$] & 0   & 5   & 10  & 15  & 20   & 25  & -5  & -10  & -15 & -20 & -25  & 30 & 35 \\ \hline
I [mV]              & 1,2 & 1,4 & 1,4 & 1,1 & 0,35 & 0,2 & 1,2 & 0,95 & 0,7 & 0,1 & 0,05 & 0  & 0  \\ \hline
\end{tabular}
\end{table}

\begin{table}[h!]
\centering
\caption{Wyniki pomiarów dla $l=0,03$ m.}
\begin{tabular}{|c|c|c|c|c|c|c|c|c|c|c|c|c|c|}
\hline
$\theta$ [$^\circ$] & 0    & 5   & 10  & 15  & 20   & 25   & 30    & -5  & -10  & -15 & -20  & -25  & -30  \\ \hline
I [mV]              & 0,25 & 0,3 & 0,3 & 0,4 & 0,25 & 0,25 & 0,125 & 0,3 & 0,35 & 0,4 & 0,15 & 0,15 & 0,15 \\ \hline
\end{tabular}
\end{table}


\begin{table}[h!]
\centering
\caption{Wyniki pomiarów dla $l=0,02$ m i $d=0,08$ m.}
\begin{tabular}{|c|c|c|c|c|c|c|c|c|c|c|c|c|c|c|c|}
\hline
$\theta$ [$^\circ$] & 0       & 1      & 2      & 3       & 4       & 5      & 6      & 7      & 8      & 9           & 10         & 11          & 12         & 13    \\ \hline
I [mV]              & 0,55    & 0,6    & 0,65   & 0,65    & 0,675   & 0,7    & 0,675  & 0,65   & 0,575  & 0,475       & 0,35       & 0,275       & 0,175      & 0,1   \\ \hline
$\theta$ [$^\circ$] & 14      & 15     & 16     & 17      & 18      & 19     & 20     & 21     & 22     & 23          & 24         & 25          & 26         & 27    \\ \hline
I [mV]              & 0,05    & 0,025  & 0,025  & 0,05    & 0,125   & 0,2    & 0,25   & 0,325  & 0,35   & 0,35        & 0,35       & 0,4         & 0,5        & 0,6   \\ \hline
$\theta$ [$^\circ$] & 28      & 29     & 30     & 31      & 32      & 33     & 34     & 35     & 36     & 37          & 38         & 39          & 40         & 41    \\ \hline
I [mV]              & 0,6     & 0,575  & 0,5    & 0,425   & 0,325   & 0,25   & 0,225  & 0,175  & 0,15   & 0,1         & 0,05       & 0,05        & 0,025      & 0,025 \\ \hline
$\theta$ [$^\circ$] & 42      & 43     & 44     & 45      & 46      & 47     & 48     & 49     & 50     & 51          & 52         & 53          & 54         & 55    \\ \hline
I [mV]              & 0,025   & 0,025  & 0,025  & 0,05    & 0,05    & 0,075  & 0,1    & 0,1    & 0,125  & 0,125       & 0,125      & 0,125       & 0,15       & 0,125 \\ \hline
$\theta$ [$^\circ$] & 56      & 57     & -1     & -2      & -3      & -4     & -5     & -6     & -7     & -8          & -9         & -10         & -11        & -12   \\ \hline
I [mV]              & 0,125   & 0,1    & 0,525  & 0,5     & 0,425   & 0,375  & 0,325  & 0,25   & 0,2    & 0,125       & 0,075      & 0,05        & 0,025      & 0,025 \\ \hline
$\theta$ [$^\circ$] & -13     & -14    & -15    & -16     & -17     & -18    & -19    & -20    & -21    & -22         & -23        & -24         & -25        & -26   \\ \hline
I [mV]              & 0,05    & 0,05   & 0,1    & 0,2     & 0,3     & 0,375  & 0,45   & 0,525  & 0,575  & 0,6         & 0,575      & 0,5         & 0,45       & 0,45  \\ \hline
$\theta$ [$^\circ$] & -27     & -28    & -29    & -30     & -31     & -32    & -33    & -34    & -35    & -36         & -37        & -38         & -39        & -40   \\ \hline
I [mV]              & 0,425   & 0,35   & 0,3    & 0,25    & 0,175   & 0,1    & 0,075  & 0,05   & 0,025  & 0,025       & 0,025      & 0,025       & 0,05       & 0,05  \\ \hline
$\theta$ [$^\circ$] & -41     & -42    & -43    & -44     & -45     & -46    & -47    & -48    & -49    & -50         & -51        & -52         & -53        & -54   \\ \hline
I [mV]              & 0,075   & 0,1    & 0,1    & 0,1     & 0,15    & 0,15   & 0,175  & 0,175  & 0,2    & 0,2         & 0,2        & 0,15        & 0,15       & 0,15  \\ \hline
$\theta$ [$^\circ$] & \multicolumn{3}{c|}{-55}  & \multicolumn{3}{c|}{-56}   & \multicolumn{3}{c|}{-57} & \multicolumn{2}{c|}{-58} & \multicolumn{2}{c|}{-59} & -60   \\ \hline
I [mV]              & \multicolumn{3}{c|}{0,15} & \multicolumn{3}{c|}{0,125} & \multicolumn{3}{c|}{0,1} & \multicolumn{2}{c|}{0,1} & \multicolumn{2}{c|}{0,1} & 0,1   \\ \hline
\end{tabular}
\end{table}


\begin{table}[h!]
\centering
\caption{Wyniki pomiarów dla $l=0,04$ m i $d=0,1$ m.}
\begin{tabular}{|c|c|c|c|c|c|c|c|c|c|c|c|c|c|c|c|}
\hline
$\theta$ [$^\circ$] & 0     & 2     & 4     & 6     & 8     & 10    & 12    & 14    & 16          & 18          & 20          & 22         & 24            & 26         \\ \hline
I [mV]              & 0,975 & 1,15  & 1,175 & 1     & 0,65  & 0,125 & 0,075 & 0,1   & 0,35        & 0,7         & 0,775       & 0,725      & 0,5           & 0,45       \\ \hline
$\theta$ [$^\circ$] & 28    & 30    & 32    & 34    & 36    & 38    & 40    & 42    & 44          & 46          & 48          & 50         & 52            & 54         \\ \hline
I [mV]              & 0,375 & 0,2   & 0,125 & 0,125 & 0,2   & 0,375 & 0,275 & 0,25  & 0,25        & 0,225       & 0,2         & 0,15       & 0,1           & 0,05       \\ \hline
$\theta$ [$^\circ$] & 56    & 58    & 60    & -2    & -4    & -6    & -8    & -10   & -12         & -14         & -16         & -18        & -20           & -22        \\ \hline
I [mV]              & 0,05  & 0,05  & 0,05  & 0,6   & 0,45  & 0,225 & 0,1   & 0,125 & 0,3         & 0,5         & 0,675       & 0,875      & 0,775         & 0,65       \\ \hline
$\theta$ [$^\circ$] & -24   & -26   & -28   & -30   & -32   & -34   & -36   & -38   & -40         & -42         & -44         & -46        & -48           & -50        \\ \hline
I [mV]              & 0,4   & 0,275 & 0,25  & 0,275 & 0,35  & 0,35  & 0,35  & 0,35  & 0,35        & 0,3         & 0,275       & 0,15       & 0,125         & 0,1        \\ \hline
$\theta$ [$^\circ$] & -52   & -54   & -56   & -58   & -60   & 1     & 3     & 5     & \multicolumn{2}{c|}{-1}   & \multicolumn{2}{c|}{-3}  & \multicolumn{2}{c|}{-5}    \\ \hline
I [mV]              & 0,075 & 0,075 & 0,075 & 0,05  & 0,075 & 1,075 & 1,175 & 1,125 & \multicolumn{2}{c|}{0,85} & \multicolumn{2}{c|}{0,6} & \multicolumn{2}{c|}{0,325} \\ \hline
\end{tabular}
\end{table}


\begin{center}
\textbf{\subsection*{ANALIZA DANYCH}}
\end{center}

Punkty z Tabel 1-4 naniesiono na wykresy przedstawione kolejno na Rysunkach 1-4.

\begin{figure}[h!]
\centering
\begin{minipage}{0.5\textwidth}
  \centering
  \includegraphics[width=8cm, height=5cm ]{rap17rys1} 
\caption{Punkty pomiarowe: Tabela 1.}
\end{minipage}%
\begin{minipage}{0.5\textwidth}
  \centering
  \includegraphics[width=8cm, height=5cm ]{rap17rys2} 
\caption{Punkty pomiarowe: Tabela 2.}
\end{minipage}
\end{figure}
\begin{figure}[h!]
\centering
\begin{minipage}{0.5\textwidth}
  \centering
  \includegraphics[width=8cm, height=5cm ]{rap17rys3} 
\caption{Punkty pomiarowe: Tabela 3.}
\end{minipage}%
\begin{minipage}{0.5\textwidth}
  \centering
  \includegraphics[width=8cm, height=5cm ]{rap17rys32} 
\caption{Punkty pomiarowe: Tabela 4.}
\end{minipage}
\end{figure}

Z otrzymanych wyników, tylko dane z Tabeli 1 i Tabeli 3 przypominają kształtem oczekiwaną krzywą. W przypadku danych z Tabeli 2 i Tabeli 4 wzięto pod uwagę możliwość odczytu wyników ze złej skali i próbowano dokonać przeliczenia wyników z jednej skali na inną. Otrzymane wyniki również były rozrzucone chaotycznie. Tak więc w dalszej analizie brano pod uwagę tylko wyniki z Tabeli 1 i Tabeli 2. 
Niepewności napięć obliczono, korzystając ze wzoru:
\begin{equation}
u_{I}=\sqrt{\dfrac{\Delta^2}{3}+(0,015I)^2},
\end{equation}
gdzie $\Delta=0,05$ mV dla Tabeli 1 i $\Delta=0,025$ mV dla danych z Tabeli 3 jest dokładnością, z jaką obserwator był w stanie odczytać wynik, a drugi człon wynika z tego, iż dokładność miernika wynosi 1,5\%.


Do danych próbowano dopasować krzywą zadaną Równaniem (1). Wykorzystano w tym celu program \textit{gnuplot}. Krzywe najlepszego dopasowania są przedstawione na Rysunku 5 i Rysunku 6. Parametry dopasowania znajdują się w Tabeli 7.

\begin{table}[h!]
\centering
\caption{Parametry dopasowania.}
\begin{tabular}{|c|c|c|c|c|}
\hline
              & \multicolumn{2}{c|}{Tabela 1} & \multicolumn{2}{c|}{Tabela 3} \\ \hline
Parametr      & Wartość      & Niepewność     & Wartość      & Niepewność     \\ \hline
$\lambda$ [m] & 20,837       & 1253           & 20,6914      & 599,6          \\ \hline
$l$ [m]       & 46,0549      & 2774           & 37,885       & 1099           \\ \hline
$I_{0}$ [mV]  & 2,41435      & 0,5299         & 1,5191       & 0,1799         \\ \hline
\end{tabular}
\end{table}


\begin{figure}[h!]
\centering
\begin{minipage}{0.5\textwidth}
  \centering
  \includegraphics[width=8cm, height=5cm ]{rap17rys12} 
\caption{Krzywa najlepszego dopasowania:\hspace{\textwidth} Tabela 1.}
\end{minipage}%
\begin{minipage}{0.5\textwidth}
  \centering
  \includegraphics[width=8cm, height=5cm ]{rap17rys322} 
\caption{Krzywa najlepszego dopasowania:\hspace{\textwidth} Tabela 3.}
\end{minipage}
\end{figure}


Wartości $\chi^2$ wynoszą kolejno 1250 i 955 dla danych z Tabeli 1 i Tabeli 3. Wartości te są znacznie większe od wartości krytycznych, więc nie można mówić o zgodności wyników z przewidywaniami teoretycznymi. 

Na Rysunku 5 i rysunku 6 widać, iż dane wydają się być przesunięte względem punktu (0,0) o stałą wartość. Po przesunięciu punktów o 2,5$^\circ$ w lewo, postanowiono jeszcze raz dopasować krzywą do punktów. Parametry nowego dopasowania znajdują się w Tabeli 8, a nowe krzywe dopasowania znajdują się na Rysunku 7 i Rysunku 8.

\begin{figure}[h!]
\centering
\begin{minipage}{0.5\textwidth}
  \centering
  \includegraphics[width=8cm, height=5cm ]{rap17rys121} 
\caption{Krzywa najlepszego dopasowania:\hspace{\textwidth} Tabela 1.}
\end{minipage}%
\begin{minipage}{0.5\textwidth}
  \centering
  \includegraphics[width=8cm, height=5cm ]{rap17rys321} 
\caption{Krzywa najlepszego dopasowania: \hspace{\textwidth}  Tabela 3.}
\end{minipage}
\end{figure}


\begin{table}[h!]
\centering
\caption{Parametry dopasowania.}
\begin{tabular}{|c|c|c|c|c|}
\hline
              & \multicolumn{2}{c|}{Tabela 1} & \multicolumn{2}{c|}{Tabela 3} \\ \hline
Parametr      & Wartość      & Niepewność     & Wartość      & Niepewność     \\ \hline
$\lambda$ [m] & 27,916       & 983,8          & 26,85        & 571,6          \\ \hline
$l$ [m]       & 64,944       & 2291           & 48,5263      & 1035           \\ \hline
$I_{0}$ [mV]  & 2,63736      & 0,2578         & 1,473        & 0,1099         \\ \hline
\end{tabular}
\end{table}

Jako, że zastosowano przesunięcie wszystkich punktów o stałą wartość, to test $\chi^2$ nie może zostać przeprowadzony. Przyglądając się rysunkom, widać, dobre dopasowanie krzywej na Rysunku 7, czego nie można powiedzieć o krzywej z Rysunku 8. Dodatkowo wartości parametrów nie zgadzają się z wartościami rzeczywistymi, szerokość szczeliny otrzymana poprzez dopasowanie krzywej jest wielokrotnie większa od wartości rzeczywistej, a niepewność tej wielkości jest zbyt duża, by móc sensownie porównywać otrzymane wartości. Można uznać, iż otrzymane dane nie są zgodne z przewidywaniami.

Dane z Tabeli 5 i Tabeli 6 przedstawiono kolejno na Rysunku 9 i Rysunku 10. Niepewności napięć obliczono z Równania (3) dla $\Delta=0,025$ mV. Do tych danych dopasowano zależność daną Równaniem (2), przy czym na potrzeby analizy danych  od razu zastosowano przesunięcie punktów o -5$^\circ$. Krzywe dopasowania są przedstawione na Rysunku 11 i Rysunku 12, a parametry dopasowania znajdują się w Tabeli 8.

\begin{figure}[h!]
\centering
\begin{minipage}{0.5\textwidth}
  \centering
  \includegraphics[width=8cm, height=5cm ]{rap17rys4} 
\caption{Punkty pomiarowe: Tabela 5.}
\end{minipage}%
\begin{minipage}{0.5\textwidth}
  \centering
  \includegraphics[width=8cm, height=5cm ]{rap17rys5} 
\caption{Punkty pomiarowe: Tabela 6.}
\end{minipage}
\end{figure}

\begin{table}[h!]
\centering
\caption{Parametry dopasowania.}
\begin{tabular}{|c|c|c|c|c|}
\hline
              & \multicolumn{2}{c|}{Tabela 5} & \multicolumn{2}{c|}{Tabela 6} \\ \hline
Parametr      & Wartość       & Niepewność    & Wartość      & Niepewność     \\ \hline
$\lambda$ [m] & 0,0343479     & 0,08395       & 10,179       & 14,3           \\ \hline
$l$ [m]       & 0,0249651     & 0,06044       & 8,6088       & 11,99          \\ \hline
$I_{0}$ [mV]  & 0,186343      & 0,009043      & 0,311253     & 0,01138        \\ \hline
$d$ [m]       & 0,0789246     & 0,1929        & 29,6492      & 41,7           \\ \hline
$\gamma$      & 0,735274      & 0,07815       & 0,73256      & 0,05756        \\ \hline
\end{tabular}
\end{table}


\begin{figure}[h!]
\centering
\begin{minipage}{0.5\textwidth}
  \centering
  \includegraphics[width=8cm, height=5cm ]{rap17rys6} 
\caption{Krzywa najlepszego dopasowania:\hspace{\textwidth} Tabela 5.}
\end{minipage}%
\begin{minipage}{0.5\textwidth}
  \centering
  \includegraphics[width=8cm, height=5cm ]{rap17rys7} 
\caption{Krzywa najlepszego dopasowania: \hspace{\textwidth}  Tabela 6.}
\end{minipage}
\end{figure}

Tak jak wcześniej, ze względu na przesunięcie punktów, przeprowadzenie testu $\chi^2$ nie jest możliwe. Pozostaje więc wizualna ocena dopasowania krzywych do punktów. Rysunek 11 z całą pewnością nie przedstawia poprawnego dopasowania krzywej do punktów, z kolei Rysunek 12 zdaje się wskazywać na przybliżoną zgodność danych z Równaniem (2) w przedziale trzech środkowych maksimów. Jednakże żadne z tych dopasowań nie jest w pełni zgodne z danymi. Po raz kolejny otrzymane dane nie są zgodne z hipotezą. 

Rysunek 13 przedstawia dane z Tabeli 5 wraz z naniesioną krzywą najlepszego dopasowania, dla której założono, iż $\lambda=0,03$ oraz $\gamma=1$. Jak widać, krzywa ta nie pokrywa się z danymi eksperymentalnymi. Po raz kolejny otrzymano niezgodność wyników. 


\begin{figure}[h!]
\includegraphics[width=10cm]{rap17rys8} 
\centering
\caption{Dopasowanie krzywej.}
\end{figure}

Szybki rzut oka na Rysunek 13 sugeruje, iż lepsze dopasowanie otrzymano by, gdyby wykres przesunąć w lewą stronę. Jednakże bliższa analiza wykresu i punków pomiarowych ujawnia asymetrię między różnicami punków i wykresu. Podczas gdy drugie maksimum z lewej strony jest bardzo dobrze dopasowane do krzywej, to jego odpowiednik po prawej stronie przejawia maksymalne odchylenie od krzywej. Tak więc przesunięcie wykresu nie naprawi dopasowania krzywej. 



\begin{center}
\textbf{\subsection*{DYSKUSJA WYNIKÓW I WNIOSKI}}
\end{center} 
Otrzymane wartości są niezgodne z przewidywaniami teoretycznymi, choć wizualnie zdają się przedstawiać poszukiwany rozkład. Prawdopodobnie na niezgodność wyników złożyło się kilka czynników: generator nie znajdował się idealnie w punkcie 0, co powodowało przesuniecie całego wykresu; szczeliny mogły nie być ustawione symetrycznie względem generatora, co zmieniało rozkład fali, mogło również dochodzić do interferencji z falami odbitymi od ścian; szczególnie widać to po wartościach w drugim maksimum, to po lewej stronie znajdowało się blisko ścin i przedstawia ono inne wartości, niż jego odpowiednik po prawej stronie, który był maksymalnie oddalony od ścian. Podsumowując: choć eksperyment nie zakończył się sukcesem, to otrzymane dane i ich analiza pozwolą na lepsze przeprowadzenie takich doświadczeń w przyszłości.




\end{document}